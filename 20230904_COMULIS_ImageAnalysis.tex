% Compile with:
% latexmk -pdf -pvc -interaction=nonstopmode *ImageAnalysis.tex
%\documentclass[aspectratio=169,draft]{beamer}
\documentclass[aspectratio=169]{beamer}

\usetheme{UniBern}

\title{\uct and image analysis}
\author{David Haberthür}
\institute{Institute of Anatomy\\University of Bern\\Switzerland}
\date{September 05, 2023 | \href{https://www.ana.unibe.ch/continuing_education/comulis_training_school/}{COMULIS Training School -- Imaging Accross Scales}}

%\includeonlyframes{current}
%then....
%\begin{frame}[label=current]
%\end{frame}

% Some often used abbreviations/commands
\newcommand{\everyframe}{100}% use only every nth frame for the animations
\newcommand{\imagewidth}{\linewidth}% set global image width
\newcommand{\imageheight}{0.618\paperheight}% set global image height
\newlength\imagescale% needed for scalebars
\newcommand{\uct}{{\textmu}CT\xspace}% make our life easier
\newcommand{\eg}{e.\,g.\xspace}%
\newcommand{\ie}{i.\,e.\xspace}%

\usepackage[backend=biber,
	style=numeric,
	backref=true,
	url=false,
	maxnames=1,
	sorting=none]{biblatex}
\addbibresource{../../Documents/library.bib}% FastSSD, Windows or Mac works (on Linux/FastSSD we generated a 'Document' folder at the correct level and `ln -s ~/P/Documents/library.bib .` to it)
\usepackage{standalone}
\usepackage{tikz}
	\usetikzlibrary{spy}
	\usetikzlibrary{backgrounds}
	\tikzset{shadowed/.style={preaction={transform canvas={shift={(1pt,-1pt)}},draw=ubRed}}}
\usepackage{shadowtext}% for the shadowed scalebar
	\shadowoffset{1pt}
	\shadowcolor{ubRed}
\usepackage{pgfplots}
	\pgfplotsset{compat=newest}
\usepackage{pgfplotstable}
\usepackage{booktabs}
\usepackage{colortbl}
\usepackage[detect-all=true,
	range-phrase=--,
	range-units=single,
	per-mode=symbol,
	per-symbol=/]{siunitx}
\usepackage[absolute,overlay]{textpos}%for the \source{} command
\usepackage[missing=main]{gitinfo2}% GitHub Actions don't pull in the commit hash, so we force the footer some lines down.
\usepackage{xspace}
\usepackage{ccicons}
\usepackage[version=4]{mhchem}
\usepackage{animate}
\usepackage{fontawesome5}
\usepackage{csquotes}
\usepackage{lipsum}%for alignment testing
\usepackage{datetime2}
\usepackage{mathastext}

% Define complementary colors to ubRed
\definecolor{ubRedComplementary1}{HTML}{00a1e6}
\definecolor{ubRedComplementary2}{HTML}{00e645}

% change tikz font to slide font
% https://tex.stackexchange.com/a/33329/828
\usepackage[eulergreek]{sansmath}
	\pgfplotsset{tick label style = {font=\sansmath\sffamily},
		every axis label = {font=\sansmath\sffamily},
		legend style = {font=\sansmath\sffamily},
		label style = {font=\sansmath\sffamily}
		}

% Globally thicker lines in with tikz
% https://tex.stackexchange.com/a/206769/828
\tikzset{every picture/.style={thick}}

% Acknowledge images just below them
% Based on https://tex.stackexchange.com/a/282637/828
\newcommand{\source}[2]{%
	% Print out (short) link under image, with small text
	\raisebox{-1.618ex}{%
		\makebox[0pt][r]{%
			\scriptsize\href{http://#1}{#1} #2%
			}%
		}%
	}%
\newcommand{\sourcecite}[2]{%
	% Cite (an image from) a reference
	\raisebox{-1.618ex}{%
		\makebox[0pt][r]{%
			\scriptsize From \cite{#1}, #2%
			}%
		}%
	}%
\newcommand{\sourcelink}[3]{%
	% Make the source command an \href{link}{text}
	\raisebox{-1.618ex}{%
		\makebox[0pt][r]{%
			\scriptsize\href{http://#1}{#2}, #3%
			}%
		}%
	}%

% Define us a custom footer *with* progress bar, based on https://tex.stackexchange.com/a/59749/828
\makeatletter
\def\progressbar@progressbar{}% the progress bar
\newcount\progressbar@tmpcounta% auxiliary counter
\newcount\progressbar@tmpcountb% auxiliary counter
\newdimen\progressbar@pbht%progressbar height
\newdimen\progressbar@pbwd%progressbar width
\newdimen\progressbar@rcircle% radius for the circle
\newdimen\progressbar@tmpdim% auxiliary dimension
\progressbar@pbwd=0.85\linewidth%
\progressbar@rcircle=1.5pt%
\def\progressbar@progressbar{%
	\progressbar@tmpcounta=\insertframenumber%
	\progressbar@tmpcountb=\inserttotalframenumber%
	\progressbar@tmpdim=\progressbar@pbwd%
	\multiply\progressbar@tmpdim by \progressbar@tmpcounta%
	\divide\progressbar@tmpdim by \progressbar@tmpcountb%
	\par%
	\begin{tikzpicture}%
		\draw[ubGrey] (0,0) -- ++ (\progressbar@pbwd,0);%
		\draw[draw=ubRed,fill=ubGrey] (\the\dimexpr\progressbar@tmpdim-\progressbar@rcircle\relax,.5\progressbar@pbht) circle (\progressbar@rcircle);%
	\end{tikzpicture}%
	\hfill bit.ly/cmls\xspace|\xspace%
	v. \href{https://github.com/habi/Talk.2023.COMULIS/commit/\gitHash}{\gitAbbrevHash}\xspace|\xspace%
	p.\xspace\insertframenumber/\inserttotalframenumber%
	\hspace*{4ex}%
	\vspace{0.5ex}%
}
\addtobeamertemplate{footline}{}%
{%
	\begin{beamercolorbox}[wd=\paperwidth,center]{green}%
		\progressbar@progressbar%
	\end{beamercolorbox}%
}%
\makeatother

% Format bibliography for beamer
% http://tex.stackexchange.com/a/10686/828
\renewbibmacro{in:}{}
% http://tex.stackexchange.com/a/13076/828
\AtEveryBibitem{%
	\clearfield{journaltitle}
	\clearfield{pages}
	\clearfield{volume}
	\clearfield{number}
	\clearname{editor}
	\clearfield{issn}
	\clearfield{year}
}
% No parentheses around the (now empty) year: https://tex.stackexchange.com/a/147537/828
\renewcommand{\bibopenparen}{\addcomma\addspace}
\renewcommand{\bibcloseparen}{\addcomma\addspace}

% Redefine \footcite based on https://tex.stackexchange.com/a/453528/828
\DeclareCiteCommand{\footcite}[\mkbibfootnote]{%
	\usebibmacro{prenote}}{%
		\printnames[family-given]{labelname}%
		\newunit%
		\printfield{doi}%
		\newunit%
		\printlabeldateextra%
	}{\addsemicolon\space}{%
		\usebibmacro{postnote}%
	}%

% Needed for non-flashing movies: https://tex.stackexchange.com/a/649450/828
\def\zeropad#1#2{%
  \ifnum1#2<1#1
    \zeropad{#1}{0#2}%
  \else%
    #2%
  \fi%
}

% Move the text down a bit
% THIS IS A BIG HACK, IT SHOULD BE FIXED IN THE TEMPLATE
\addtobeamertemplate{frametitle}{}{\vspace*{0.75em}}

\begin{document}
% No footline on the title page
% http://tex.stackexchange.com/a/18829/828 helps us to achieve that
{%
	\setbeamertemplate{footline}{}%
	\begin{frame}%
		\maketitle
	\end{frame}%
}

\begin{frame}
	\frametitle{\uct-group}
	\begin{columns}
		\begin{column}{0.61\linewidth}
			\begin{itemize}
				\item microangioCT~\cite{Hlushchuk2018}
				\begin{itemize}
					\item Angiogenesis: heart, musculature~\cite{Nording2021} and bones
					\item Vasculature: (mouse) brain~\cite{Hlushchuk2020}, (human) nerve scaffolds~\cite{Wuthrich2020}, (human) skin flaps~\cite{Zubler2021} and tumors
				\end{itemize}
				\item Zebrafish musculature and gills~\cite{MesserliAaldijk2020}
				\item (Lung) tumor detection and metastasis classification~\cite{Trappetti2021}
				\item Collaborations with museums~\cite{Bochud2021} and scientist at UniBe~\cite{Halm2021} to scan a wide range of specimens
				\item Automate \emph{all} the things!~\cite{Haberthuer2021, Haberthuer2023}
			\end{itemize}
		\end{column}
		\begin{column}{0.37\linewidth}
			\centering
			\includegraphics<1>[width=\imagewidth]{./images/1172}%
			\only<1>{\source{brukersupport.com}{}}
			\includegraphics<2>[width=\imagewidth]{./images/1272}%
			\only<2>{\source{bruker.com/skyscan1272}{}}
			\includegraphics<3>[width=\imagewidth]{./images/2214}%
			\only<3>{\source{bruker.com/skyscan2214}{}}
		\end{column}
	\end{columns}
\end{frame}

\begin{frame}
	\frametitle{Overview}
	\centering
	\includegraphics[height=\imheight]{./images/MIC-AM_techniques}%
	\sourcelink{https://anatomie.unibe.ch/tschanz}{Stefan Tschanz}{with permission}%
\end{frame}

\renewcommand{\imagewidth}{\columnwidth}
\begin{frame}
	\frametitle{History}
	\begin{columns}
		\begin{column}{0.49\linewidth}
			\begin{itemize}
				\item Long history
				\begin{itemize}
					\item \citeyear{Cormack1963}:~\citeauthor{Cormack1963} used a collimated \ce{^{60}Co} source and a Geiger counter as a detector~\cite{Cormack1963}%
					\item 1976:~\citeauthor{Hounsfield1976a} worked on first clinical scanner~\cite{Hounsfield1976a}%
					\item Nice overview by \citeauthor{Hsieh2003}~\cite{Hsieh2003}%
				\end{itemize}%
				\item<2-|handout:2-> CT scanner generations: First\uncover<3-|handout:3->{, second}\uncover<4|handout:4->{ and third}%
			\end{itemize}
		\end{column}
		\begin{column}{0.49\linewidth}
			\centering
			\includegraphics<1|handout:1>[height=\imageheight]{./images/ch8f2}%
			\only<1|handout:1>{\sourcecite{Taubmann2018}{Figure 8.2}}%
			\includegraphics<2|handout:2>[width=\imagewidth]{./images/History_Generation1}%
			\only<2|handout:2>{\sourcecite{Hsieh2003}{Figure 1.12}}%
			\includegraphics<3|handout:3>[width=\imagewidth]{./images/History_Generation2}%
			\only<3|handout:3>{\sourcecite{Hsieh2003}{Figure 1.13}}%
			\includegraphics<4|handout:4>[width=\imagewidth]{./images/History_Generation3}%
			\only<4|handout:4>{\sourcecite{Hsieh2003}{Figure 1.14}}%
		\end{column}
	\end{columns}
\end{frame}

\subsection{Interaction of x-rays with matter}
\begin{frame}
	\frametitle{X-ray interaction}
	\begin{itemize}
		\item \textquote[\cite{xrayphysics}]{X-rays interact with tissue in 2 main ways: photoelectric effect and Compton scatter.
				To a first approximation, the photoelectric effect contributes to contrast while the Compton effect contributes to noise.
				Both contribute to dose.}
		\begin{itemize}
			\item Photoelectric absorption (\(\tau\)) is strongly dependent on the atomic number \(Z\) of the absorbing material: \(\tau\propto\frac{Z^4}{E^{3.5}}\)
			\note{From \href{https://radiopaedia.org/articles/photoelectric-effect}{Radiopaedia.org}: Therefore if \(Z\) doubles, PEA will increase by a factor of 16 (\(2^4=16\)), and if \(E\) doubles PEA will be reduced by a factor of 11.
				Small changes in \(Z\) and \(E\) can therefore significantly affect PEA.
				This has practical implications in the field of radiation protection and is the reason why materials with a high \(Z\) such as lead (\(Z= 82\)) are useful shielding materials.
				The dependence of PEA on \(Z\) and \(E\) means that it is the major contributor to beam attenuation up to approximately 30 keV when human tissues (\(Z=7.4\)) are irradiated.
				At beam energies above this, the Compton effect predominates.}
			\item Compton scattering is one of the principle forms of photon interaction and is directly proportional to the (electron \& physical) density of the material.
				It does \emph{not} depend on the atomic number: \(\lambda' - \lambda = \frac{h}{m_e c}\left(1-\cos{\theta}\right)\)
			\note{Where \(\lambda\) is the initial wavelength, \(\lambda'\) is the wavelength after scattering, \(h\) is the Planck constant, \(m_e\) is the electron rest mass, \(c\) is the speed of light, and \(\theta\) is the scattering angle.}
		\end{itemize}
		\item Lowering x-ray energy increases contrast
		\item X-ray penetration decreases exponentially with sample thickness (\cite[\ie Beer-Lamberts law]{wiki:beer-lambert} \(I(t) = I_0 \, e^{-\alpha z}\)
	\end{itemize}
\end{frame}

\begin{frame}
	\frametitle{Composition of biological tissues}
	Tissue: content by mass percentage
	\centering
	\begin{table}%
		\only<1>{%
			\pgfplotstabletypeset[%
				col sep=comma,% the seperator in our .csv file
				display columns/0/.style={string type},%
				% section 3.2 in pgfplotstable manual
				every head row/.style={before row={\toprule}},%
				every row no 0/.style={after row=\midrule},%
				every last row/.style={after row=\bottomrule},%
			]{./tables/tissue-composition.csv}%
		}%
		\only<2|handout:0>{%
			\pgfplotstabletypeset[%
				col sep=comma,% the seperator in our .csv file
				display columns/0/.style={string type},%
				every head row/.style={before row={\toprule}},%
				every row no 0/.style={after row=\midrule},%
				every last row/.style={after row=\bottomrule},%
				% make certain cells ubRed, based on https://tex.stackexchange.com/a/296914/828
				every row 1 column 2/.style={postproc cell content/.append style={/pgfplots/table/@cell content/.add={\cellcolor{ubRed!61.8}}{}}},%
				every row 1 column 4/.style={postproc cell content/.append style={/pgfplots/table/@cell content/.add={\cellcolor{ubRed!61.8}}{}}},%
				every row 6 column 6/.style={postproc cell content/.append style={/pgfplots/table/@cell content/.add={\cellcolor{ubRed!61.8}}{}}},%
				every row 6 column 10/.style={postproc cell content/.append style={/pgfplots/table/@cell content/.add={\cellcolor{ubRed!61.8}}{}}}%
			]{./tables/tissue-composition.csv}%
		}%
	\end{table}
	\note{Bone, lean tissue, fat and air can be distinguished quite easily}
\end{frame}

\renewcommand{\imagewidth}{\columnwidth}
\begin{frame}
	\frametitle{Why \uct?}
	\begin{columns}
		\begin{column}{0.49\linewidth}
			% https://www.cancerimagingarchive.net/nbia-search/?saved-cart=nbia-76761575299081509
			\only<1-4|handout:1-4>{%
				\pgfmathsetlength{\imagewidth}{\imagewidth}%
				\pgfmathsetlength{\imagescale}{\imagewidth/512}%
				\def\x{316}% scalebar-x starting at golden ratio of image width of 512px = 316
				\def\y{361}% scalebar-y at 90% of image height of 401px = 361
				\begin{tikzpicture}[x=\imagescale,y=-\imagescale]
					\node[anchor=north west, inner sep=0pt, outer sep=0pt] at (0,0) {\includegraphics[width=\imagewidth]{./images/comparison/MAX_human}};
					% 512.000px = 250.0096mm -> 100px = 48830.000um -> 1.024px = 500um, 0.205px = 100um
					%\draw[|-|,blue,thick] (0,200) -- (512,200) node [sloped,midway,above,fill=white,semitransparent,text opacity=1] {\SI{250.0096}{\milli\meter} (512px) TEMPORARY!};
					\draw[|-|,white,shadowed] (\x,\y) -- (\x+102.4,\y) node [midway,above] {\shadowtext{\SI{5}{\centi\meter}}};
				\end{tikzpicture}%
			}%
			\only<5|handout:0>{%
				\pgfmathsetlength{\imagewidth}{\imagewidth}%
				\pgfmathsetlength{\imagescale}{\imagewidth/512}%
				\def\x{316}% scalebar-x starting at golden ratio of image width of 512px = 316
				\def\y{361}% scalebar-y at 90% of image height of 401px = 361
				\def\mag{5}% magnification of inset
				\def\size{100}% size of inset
				\begin{tikzpicture}[x=\imagescale,y=-\imagescale,spy using outlines={rectangle,magnification=\mag,size=\size,connect spies}]
					\node[anchor=north west, inner sep=0pt, outer sep=0pt] at (0,0) {\includegraphics[width=\imagewidth]{./images/comparison/MAX_human}};
					\spy [red] on (102,342) in node at (256,201) [anchor=center];
					% 512.000px = 250.0096mm -> 100px = 48830.000um -> 1.024px = 500um, 0.205px = 100um
					\draw[|-|,white,shadowed] (\x,\y) -- (\x+102.4,\y) node [midway,above] {\shadowtext{\SI{5}{\centi\meter}}};
				\end{tikzpicture}%
			}%
			\renewcommand{\imagewidth}{0.1554\columnwidth}%
			\only<6|handout:5>{%
				\centering
				\pgfmathsetlength{\imagewidth}{\imagewidth}%
				\pgfmathsetlength{\imagescale}{\imagewidth/512}%
				\def\x{316}% scalebar-x starting at golden ratio of image width of 512px = 316
				\def\y{361}% scalebar-y at 90% of image height of 401px = 361
				\begin{tikzpicture}[x=\imagescale,y=-\imagescale]
					\node[anchor=north west, inner sep=0pt, outer sep=0pt] at (0,0) {\includegraphics[width=\imagewidth]{./images/comparison/MAX_human}};
					% 512.000px = 250.0096mm -> 100px = 48830.000um -> 1.024px = 500um, 0.205px = 100um
					\draw[|-|,white,shadowed] (\x,\y) -- (\x+102.4,\y) node [midway,above] {\shadowtext{\SI{5}{\centi\meter}}};
				\end{tikzpicture}%
			}%
			\sourcecite{Clark2013}{Subject \emph{C3L-02465}}
		\end{column}%
		\begin{column}{0.49\linewidth}
			\only<1|handout:1>{%
				\pgfmathsetlength{\imagewidth}{\imagewidth}%
				\pgfmathsetlength{\imagescale}{\imagewidth/3295}%
				\def\x{2036}% scalebar-x starting at golden ratio of image width of 3295px = 2036
				\def\y{1343}% scalebar-y at 90% of image height of 1492px = 1343
				\begin{tikzpicture}[x=\imagescale,y=-\imagescale]
					\node[anchor=north west, inner sep=0pt, outer sep=0pt] at (0,0) {\includegraphics[width=\imagewidth]{./images/comparison/MAX_mouse}};
					% 3295.000px = 26.2282mm -> 100px = 796.000um -> 62.814px = 500um, 12.563px = 100um
					%\draw[|-|,blue,thick] (0,746) -- (3295,746) node [sloped,midway,above,fill=white,semitransparent,text opacity=1] {\SI{26.2282}{\milli\meter} (3295px) TEMPORARY!};
					\draw[|-|,white,shadowed] (\x,\y) -- (\x+628.14,\y) node [midway,above] {\shadowtext{\SI{5}{\milli\meter}}};
				\end{tikzpicture}%
			}%
			\renewcommand{\imagewidth}{0.1\columnwidth}%
			\only<2|handout:2>{%
				\centering
				\pgfmathsetlength{\imagewidth}{\imagewidth}%
				\pgfmathsetlength{\imagescale}{\imagewidth/54}%
				\def\x{33}% scalebar-x starting at golden ratio of image width of 54px = 33
				\def\y{22}% scalebar-y at 90% of image height of 24px = 22
				\begin{tikzpicture}[x=\imagescale,y=-\imagescale]
					\node[anchor=north west, inner sep=0pt, outer sep=0pt] at (0,0) {\includegraphics[width=\imagewidth]{./images/comparison/MAX_mouse_488umppx}};
					% 54.000px = 26.3682mm -> 100px = 48830.000um -> 1.024px = 500um, 0.205px = 100um
					%\draw[|-|,blue,thick] (0,12) -- (54,12) node [sloped,midway,above,fill=white,semitransparent,text opacity=1] {\SI{26.3682}{\milli\meter} (54px) TEMPORARY!};
					\draw[|-|,white,shadowed] (\x,\y) -- (\x+102.4,\y) node [midway,above] {\shadowtext{\SI{5}{\centi\meter}}};
					%\draw[color=red, anchor=south west] (0,24) node [fill=white, semitransparent] {Legend} node {Legend};
				\end{tikzpicture}%
			}%
			\renewcommand{\imagewidth}{\columnwidth}
			\only<3|handout:3>{%
				\centering
				\pgfmathsetlength{\imagewidth}{\imagewidth}%
				\pgfmathsetlength{\imagescale}{\imagewidth/54}%
				\def\x{33}% scalebar-x starting at golden ratio of image width of 54px = 33
				\def\y{22}% scalebar-y at 90% of image height of 24px = 22
				\begin{tikzpicture}[x=\imagescale,y=-\imagescale]
					\node[anchor=north west, inner sep=0pt, outer sep=0pt] at (0,0) {\includegraphics[width=\imagewidth]{./images/comparison/MAX_mouse_488umppx}};
					% 54.000px = 26.3682mm -> 100px = 48830.000um -> 1.024px = 500um, 0.205px = 100um
					%\draw[|-|,blue,thick] (0,12) -- (54,12) node [sloped,midway,above,fill=white,semitransparent,text opacity=1] {\SI{26.3682}{\milli\meter} (54px) TEMPORARY!};
					\draw[|-|,white,shadowed] (\x,\y) -- (\x+10.24,\y) node [midway,above] {\shadowtext{\SI{5}{\milli\meter}}};
					%\draw[color=red, anchor=south west] (0,24) node [fill=white, semitransparent] {Legend} node {Legend};
				\end{tikzpicture}%
			}%
			\only<4|handout:4>{%
				\pgfmathsetlength{\imagewidth}{\imagewidth}%
				\pgfmathsetlength{\imagescale}{\imagewidth/3295}%
				\def\x{2036}% scalebar-x starting at golden ratio of image width of 3295px = 2036
				\def\y{1343}% scalebar-y at 90% of image height of 1492px = 1343
				\begin{tikzpicture}[x=\imagescale,y=-\imagescale]
					\node[anchor=north west, inner sep=0pt, outer sep=0pt] at (0,0) {\includegraphics[width=\imagewidth]{./images/comparison/MAX_mouse}};
					% 3295.000px = 26.2282mm -> 100px = 796.000um -> 62.814px = 500um, 12.563px = 100um
					\draw[|-|,white,shadowed] (\x,\y) -- (\x+628.14,\y) node [midway,above] {\shadowtext{\SI{5}{\milli\meter}}};
				\end{tikzpicture}%
			}%
			\only<5|handout:0>{%
				\pgfmathsetlength{\imagewidth}{\imagewidth}%
				\pgfmathsetlength{\imagescale}{\imagewidth/3295}%
				\def\x{2036}% scalebar-x starting at golden ratio of image width of 3295px = 2036
				\def\y{1343}% scalebar-y at 90% of image height of 1492px = 1343
				\def\mag{5}% magnification of inset
				\def\size{100}% size of inset
				\begin{tikzpicture}[x=\imagescale,y=-\imagescale,spy using outlines={rectangle,magnification=\mag,size=\size,connect spies}]
					\node[anchor=north west, inner sep=0pt, outer sep=0pt] at (0,0) {\includegraphics[width=\imagewidth]{./images/comparison/MAX_mouse}};
					\spy [red] on (352,1116) in node at (1648,746) [anchor=center];
					% 3295.000px = 26.2282mm -> 100px = 796.000um -> 62.814px = 500um, 12.563px = 100um
					\draw[|-|,white,shadowed] (\x,\y) -- (\x+628.14,\y) node [midway,above] {\shadowtext{\SI{5}{\milli\meter}}};
				\end{tikzpicture}%
			}%
			\only<6|handout:5>{%
				\pgfmathsetlength{\imagewidth}{\imagewidth}%
				\pgfmathsetlength{\imagescale}{\imagewidth/3295}%
				\def\x{2036}% scalebar-x starting at golden ratio of image width of 3295px = 2036
				\def\y{1343}% scalebar-y at 90% of image height of 1492px = 1343
				\begin{tikzpicture}[x=\imagescale,y=-\imagescale]
					\node[anchor=north west, inner sep=0pt, outer sep=0pt] at (0,0) {\includegraphics[width=\imagewidth]{./images/comparison/MAX_mouse}};
					% 3295.000px = 26.2282mm -> 100px = 796.000um -> 62.814px = 500um, 12.563px = 100um
					\draw[|-|,white,shadowed] (\x,\y) -- (\x+628.14,\y) node [midway,above] {\shadowtext{\SI{5}{\milli\meter}}};
				\end{tikzpicture}%
			}%
		\end{column}
	\end{columns}
\end{frame}
\note{The human head scan was downloaded from the \href{https://www.cancerimagingarchive.net}{Cancer Imaging Archive}.
	We loaded the DICOM slices in Fiji, resliced it to show it from the side and then used to generate an MIP.
	According to the DICOM tags, the voxel size is \SI{0.4883x0.4883x0.625}{\milli\meter\cubed}, the image size is 512\(\times\)512 pixels.
	The mouse head is the same as shown in the early animation.
	The files from the early animation were resized 0.25 times; here we used the original dataset (Mouse1265\_Skull\_Gaby\_TKI\_7\_96um\_Al05\_2K) for a reslice and the generation of the MIP.
	The voxel size of the original data is 7.96 um, the image size is 3295\(\times\)1492 pixels.
	}

\section{Data wrangling by example}
\begin{frame}
	\begin{itemize}
		\item Cichlids
		\item Image processing
		\item Python/Jupyter
	\end{itemize}
\end{frame}

\begin{frame}
	\frametitle{Data wrangling by example}
	\centering
	\includegraphics<1>[height=\imageheight]{./images/cichlids/Otolither_104016_head_01_Overview}%
	\includegraphics<2>[height=\imageheight]{./images/cichlids/Otolither_104016_head_02_GrayValues}%
	\includegraphics<3>[height=\imageheight]{./images/cichlids/Otolither_104016_head_03_GrayValuesRegion}%
	\includegraphics<4>[height=\imageheight]{./images/cichlids/Otolither_104016_head_04_GrayValuesSmoothed}%
	\includegraphics<5>[height=\imageheight]{./images/cichlids/Otolither_104016_head_05_Peaks}%
	\includegraphics<6>[height=\imageheight]{./images/cichlids/Otolither_104016_head_06_Peaks_All}%
	\includegraphics<7>[height=\imageheight]{./images/cichlids/Otolither_104016_head_07_ExtractedRegions}%
	\includegraphics<8>[height=\imageheight]{./images/cichlids/Otolither_104016_head_08_ExtractedRegionsMIPs}%
	\includegraphics<9>[height=\imageheight]{./images/cichlids/Otolither_104016_head_09_ExtractedOtolithMasked}%
	\only<10>{\href{file:///./images/cichlids/Otolither\_104016\_head\_3D.html}{Exported 3D view}}%
\end{frame}

\section{Take home message}
\begin{frame}
	\begin{itemize}
		\item Ask me anything
		\item Let's do this in the \emph{Hands-on} session
	\end{itemize}
\end{frame}

\begin{frame}[allowframebreaks]
	\frametitle{References}%
	\renewcommand*{\bibfont}{\scriptsize}%
	\setbeamertemplate{bibliography item}{\insertbiblabel}%
	\printbibliography%
\end{frame}

\end{document}
